\HeaderA{RBFMeasure}{Calculate cosine similarity distance for a give timeseries measurement.}{RBFMeasure}
\begin{Description}\relax
Given a set of measurements and their timestamps, this function outputs a symmetric matrix that capture the pair wise similarity between every pair of timestamps.
\end{Description}
\begin{Usage}
\begin{verbatim}
RBFMeasure(X, sig)
\end{verbatim}
\end{Usage}
\begin{Arguments}
\begin{ldescription}
\item[\code{X}] X is a matrix, consisting of \$p\$ real valued measurements in each column (timestamp).

\item[\code{sig}] sigma, the spread width of the RBF kernel.

\end{ldescription}
\end{Arguments}
\begin{Value}
A non-negative symmetric matrix is returned.
\end{Value}
\begin{Author}\relax
Sriram Lakshminarasimhan        
Siddarth Ramaswamy
\end{Author}
\begin{References}\relax
Haibin Cheng, Pang-Ning Tan, Christopher Potter, Steven Klooster, "A Robust Graph-Based Algorithm for Detection and Characterization of Anomalies in Noisy Multivariate Time Series," icdmw, pp.349-358, 2008 IEEE International Conference on Data Mining Workshops, 2008
\end{References}
\begin{SeeAlso}\relax
Other random walk algorithms 'outlierA', 'outlierB'
\end{SeeAlso}
\begin{Examples}
\begin{ExampleCode}
data(KernelData)
TimeSeries = as.matrix(KernelData)

TimeSeries
RBFMeasure(TimeSeries, 1.414)
\end{ExampleCode}
\end{Examples}

