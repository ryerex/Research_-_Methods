\HeaderA{kernelAlign}{KernelAlignment of time-series measurements}{kernelAlign}
\aliasA{KAlign}{kernelAlign}{KAlign}
\begin{Description}\relax
Performs KernelAlignment algorithm described in the referenced paper, for a set of multivariate time series predictor values and a set of target values.
\end{Description}
\begin{Usage}
\begin{verbatim}
kernelAlign(X, Y, r = 0, sig = 0.1, nv = 1)
\end{verbatim}
\end{Usage}
\begin{Arguments}
\begin{ldescription}
\item[\code{X}] The predictor time series variables, each row is a variable and each column is a timestamp.

\item[\code{Y}] The target time series.

\item[\code{r}] The autoregression lag.

\item[\code{sig}] The spread width of the RBF kernel.

\item[\code{nv}] Number of eigenvectors to be used.

\end{ldescription}
\end{Arguments}
\begin{Author}\relax
Sriram Lakshminarasimhan
Siddarth Ramaswamy
\end{Author}
\begin{References}\relax
[1] Haibin Cheng, Pang-Ning Tan, Christopher Potter, and Steven Klooster. 
A Robust Graph-based Algorithm for Detection and Characterization of Anomalies in Noisy
Multivariate Time Series, to appear in Proc.of IEEE International Conference on Data Mining
workshop on Spatial and Spatio-temporal Data Mining (ICDM/STDM 08)
\end{References}
\begin{Examples}
\begin{ExampleCode}
#X is the predictor, having 2 variables whose values have been measured from
#t1 to t5 . Y is the target variable.
data(KernelPredictorSample1)
data(KernelTargetSample1)
X = as.matrix(KernelPredictorSample1)
Y = as.matrix(KernelTargetSample1)
X
Y
#kernelAlign function takes in the target and predictor as input
#along with number of eigen values, the standard deviation
#damping factor is an optional parameter
RandomWalkOutput = kernelAlign(X, Y, nv = 2)
RandomWalkOutput
\end{ExampleCode}
\end{Examples}

